%!TEX encoding =  IsoLatin
\documentclass[french,11pt]{article}
\usepackage[french]{babel}
\usepackage{a4}
\usepackage[T1]{fontenc}
\usepackage{amsmath}
\usepackage{amssymb}
\usepackage{subfigure}
\usepackage{float}
\usepackage{latexsym}
\usepackage{amsfonts}
\usepackage{epic}
\usepackage{eepic}
\usepackage{makeidx}
\usepackage{multido}
\usepackage{varindex}
\usepackage{alltt}
\usepackage{moreverb}
\newenvironment{slide}[1] {\section{#1}} {}


\begin{document}

\begin{slide}{Petits objets g�om�triques et statistiques}

\begin{itemize}
\item g�om�trie
 
     \begin{itemize}
      \item point ou vecteur :  $class$ $Point$
 
            \begin{itemize}
                \item deux coordonn�es $x$ et $y$
                \item m�thodes pour ajouter, multiplier des vecteurs, calculer une norme, un produit scalaire, un angle (ou orientation)
              \end{itemize}
              
     \item segment : $class$ $Segment$
     
            \begin{itemize}
                \item deux extr�mit�s $a$ et $b$ de type $Point$
                \item m�thodes pour calculer un vecteur directeur et un vecteur normal
              \end{itemize}
              
     \end{itemize}
     
\item statistique : calcul des probabilit�s de lois binomiales $B(n,p)$  ($fonction $  $tabule\_queue\_binom (n,p)$)

     \begin{itemize}
     \item pour $n$ aussi grand que la plus grande dimension de l'image
     \item pour $p$ : probabilit� que l'angle entre deux vecteurs < $\frac{2\pi}{32} =  \frac{1}{16}$
     \item calcul par r�currence : $a^k_m =  p a^{k-1}_{m-1} + (1-p) a ^k_{m-1 }$ 
     
     		o� $a^k_m = P(X = k)$ et $X$ suit une loi binomiale $B(m,p)$
     \end{itemize}

\end{itemize}

\end{slide}


\begin{slide}{Fonctionnement du programme informatique : trois �tapes}
  \begin{enumerate}
     \item Pr�liminaires
     \begin{itemize}
        \item chargement de l'image
        \item affichage de l'image
        \item calcul du gradient : matrice de vecteurs, $grad [ (x,y) ] = $ gradient au pixel $(x,y)$, le gradient est la seule information manipul�e par l'algorithme  % premi�re image de dessin.ppt

\begin{verbatim}        
def calcule_gradient (image) :
    """retourne le gradient d'une image sous forme d'une matrice
    de Point, consideres ici comme des vecteurs"""
    size     = X,Y = image.get_size ()
    m        = [  [ GEO.Point (0,0) for i in xrange (0,Y) ] for j in xrange (0,X) ]
    res      = Numeric.array (m)  # pour voir d'autres types, le type array du package Numeric, 
                                  # plus efficace qu'une liste de listes pour les matrices
 
    for x in xrange (0,size [0] - 1) :
        for y in xrange (0, size [1] - 1) :
            ij = image.get_at ( (x,y) ) [0]     # c'est une image en niveau de gris
            Ij = image.get_at ( (x+1,y) ) [0]   # les trois intensites sont egales
            iJ = image.get_at ( (x,y+1) )  [0]  # on ne prend que la premiere
            IJ = image.get_at ( (x+1,y+1) )  [0]
            gx = 0.5 * (IJ - iJ + Ij - ij)
            gy = 0.5 * (IJ - Ij + iJ - ij)
            res [ (x,y) ] = GEO.Point (gx,gy)            
    return res
\end{verbatim}

      \end{itemize}
     \item Algorithme de d�tection de segments
     \item Affichage du r�sultat : une image sur laquelle on superpose les segments d�tect�s
\end{enumerate}
\end{slide}

\begin{slide}{Algorithme de d�tection de segments}

\begin{enumerate}
\item choix d'un segment  : $class$ $SegmentBord$ % seconde image de dessin.ppt
\item est-ce un segment significatif ?
  \begin{enumerate}  
  \item pour chaque pixel :  le gradient est-il presque perpendiculaire au segment $\rightarrow$ 0 ou 1 % image que tu as faite
  
  		segment $\longrightarrow$ liste de 0 ou 1 (exemple : $[0,1,0,0,1,1,1,1,0,0,1,0]$, stock�s dans une instance de la classe $LigneGradient$)
		
		$class$ $SegmentBord\_Commun$, m�thode $decoupe\_gradient$
  \item  y a -t-il suffisamment de 1 pour en faire un segment ou est-ce le hasard ?
  
  		$\longrightarrow$ fait intervenir les probabilit�s
		
		$class$ $LigneGradient$, m�thode $segments\_significatifs$
  \item on r�it�re ce proc�d� pour chaque sous-liste
  \end{enumerate}
\item on recommence les �tapes pr�c�dentes pour chaque segment reliant deux bords diff�rents de l'image
\end{enumerate}

\end{slide}


\begin{slide}{Les segments reliant deux bords diff�rents de l'image}
\begin{itemize}
\item 2 param�tres
\begin{enumerate}
\item une orientation : un angle choisi dans l'ensemble $\{0,\delta,2\delta, 3\delta, ..., 2 \pi - \delta \}$
\item 1 extr�mit� : tout  pixel du contour 
\end{enumerate}
\item Comment les �tudier tous ?
\begin{enumerate}
\item choisir un premier point du contour $a$ et un angle $\alpha$ nul  
\item d�tecter si ce segment ou une sous-partie est significatif dans l'image
\item on passe au segment suivant : (next)   
\begin{itemize}
\item on change la premi�re extr�mit� : on passe au pixel suivant, on d�duit la seconde avec l'orientation
\item si on a fait le tour, on change l'orientation
\item si on a aussi fait le tour, on a fini
\end{itemize}
\end{enumerate}
\end{itemize}
\end{slide}




\end{document}

